
\keywords{%
  системы типов,
  вывод типов,
  качество по,
  статический анализ,
  интегрированные среды разработки
}

\abstractcontent{
  Точное определение типов в программах на диначически-типизированных языках
  является сложной проблемой, решение которой может быть использовано во многих
  областях, включающих улучшение поддержки этих языков в средах разработки
  (IDE). Однако существующие алгоритмы для вывода типов в языках со статической
  типизацией, а также решения, применяемые в статических анализаторах,
  неприменимы к высокоуровневым динамически типизированным языкам, таким как
  Python, или не подходят для использования в составе IDE из-за низкой точности
  результатов и/или быстродействия. 
  
  В результате использования ``утиной'' типизации (\emph{duck-typing}) и
  параметрического полиморфизма одной из областей, особенно сложных для анализа,
  является вывод типов параметров функций.  Мы предлагаем алгоритм для
  определния типов параметров функций, основанный на структурной эквивалентности
  типов и информации об использовнии значений параметров в теле функции.  Нами
  был разработан прототип, который показал эффективность предлагаемого решения
  на нескольких десятках проектов с открытыми исходными текстами. Также в работе
  обсуждаются существующие подходы к выводу типов в динамически типизированных
  языках и направления для дальнейшего развития описываемого метода.
}

\keywordsen{
  type systems,
  type inference,
  quality of software,
  static analysis,
  integrated development environments
}

\abstractcontenten{
  Precise type inference in programs written in dynamically typed languages is
  known to be a difficult problem. Its solution can be used in a lot of areas
  including improvement of support for these languages in integrated development
  environments (IDE). Nonetheless existing algorithms for type inference in
  statically typed languages as well approaches used in static analysis are
  either not suitable for high-level dynamically typed languages like Python or
  cannot be used in IDE for these languages due to low precision and/or
  performance.

  Because of the use of duck-typing and parametric polymorphism one the hardest
  problems in analysis of dynamic languages is type inference of function
  parameters. We propose an algorithm for type inference of function parameters
  based on the notion of structural equivalence of types and information about
  the use of parameter values in function body.  We've developed a prototype
  that shows effectiveness of suggested approach for several open source
  projects. We also discuss existing approaches for type inference in
  dynamically typed languages and directions of future improvement of the
  described method.
}
