%%%%%%%%%%%%%%%%%%%%%%%%%%%%%%%%%%%%%%%%%%%%%%%%%%%%%%%%%%%%%%%%%%%%%%%%%%%%%%%%
\chapter{Проектирование}
%%%%%%%%%%%%%%%%%%%%%%%%%%%%%%%%%%%%%%%%%%%%%%%%%%%%%%%%%%%%%%%%%%%%%%%%%%%%%%%%

\section{Описание алгоритмов}

\subsection{Cartesian product algorithm}
\label{sub:CPA}

Ole Ageseen, один из разработчиков языка SELF в 1995 предложил алгоритм
Cartesian Product Algorithm (CPA)~\cite{Agesen1995}, позволяющий выводить
конкретные типы аргументов для полиморфных функций. Достигается это следующим
образом.

Сначала собирается вся первоначально доступная информация о типах выражений и
переменных в программе и распространяется посредством анализа графа потока
данных (Data Flow Graph --- DFG).  Затем для каждого случая вызова функции
информация о возможных типах аргументов используется для вывода типа ее
возвращаемого значения. Причем поскольку из-за динамической типизации аргумент
может принимать значение одного из нескольких возможных типов, собирается
декартово произведение множеств возможных типов аргументов и анализ происходит
для каждого кортежа в отдельности (отсюда название алгоритма). Каждая из
обнаруженных комбинаций типов аргументов анализируется один раз.

Как отмечается в работе~\cite{Madsen2007} работы O. Agessen не раскрывают
многих аспектов практического применения алгоритма, к тому же целевой язык SELF
имеет ряд отличий от современных высокоуровневых динамически типизированных
языков таких как Ruby или Python. Тем не менее с момента его создания CPA
применялся в ряде различных проектов~\cite{Madsen2007,Salib2004,Hanov},
посвященных выводу типов в языках с динамической типизацией, и зарекомендовал
себя как достаточно точное и эффективное решение.
