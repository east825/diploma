%%%%%%%%%%%%%%%%%%%%%%%%%%%%%%%%%%%%%%%%%%%%%%%%%%%%%%%%%%%%%%%%%%%%%%%%%%%%%%%%
\conclusion
%%%%%%%%%%%%%%%%%%%%%%%%%%%%%%%%%%%%%%%%%%%%%%%%%%%%%%%%%%%%%%%%%%%%%%%%%%%%%%%%

Нами был предложен метод для вывода типов параметров функций в программах на
Python, использующий информацию об атрибутах параметров, к которым происходит
обращение, для подбора подходящих по интефейсу классов. 

Статистика, собранная при помощи разработанного прототипа, а также сохранение
требования инкрементальности обновления индексов позволяют говорить о
применимости подхода для использования в составе сред разработки. 

Проведенные эксперименты позволили оценить такие характеристики эффективности
метода, как точность (82\%) и полнота (6.24\% от общего числа проанализированных
параметров).  Низкое значение полноты не критично, потому что метод покрывает
случаи, в которых типы не могут быть выведены имеющимися средствами среды
PyCharm, для которой создавался данный подход.

В процессе оценки характеристик решения, также было найдено множество потенциальных
направлений для дальнейшего улучшения предлагаемого метода, в частности для
повышения полноты анализа. Большая их часть, однако, требует учета при анализе
потоков управления и данных в программе, что по причине ограничений использованных
при разработке компонентов не было сделано в прототипе, но осуществимо уже
на базе среды разработки PyCharm. Кроме того, совмещение предложенного нами
подхода с уже имеющимися возможностями по выводу типов в PyCharm также должно
значительно повысить качество получаемых результатов.

% \begin{enumerate}
    % \item Использование только атрибутов, к которым происходит обращение на всех
      % путях потока исполнения в теле функции.
    % \item Учитывание псевдонимов (\emph{aliases}) параметров функций.
    % \item Использовании информации о структурных типах параметров для вызываемых
      % функций.
% \end{enumerate}


Между тем, некоторая функциональность среды разработки, менее чувствительная к
низкой полноте анализа, такая как автодополнение, может быть улучшена уже при
помощи текущей версии алгоритма.

Областями дальнейшего развития идеи, также представляющими интерес, но не
рассмотренными в данной работе, являются применения алгоритма для вывода типов
локальных переменных, помимо параметров, а также использование его с другими динамически
типизированными языками, такими как Ruby и JavaScript.

Мы считаем, что применение и дальнейшее развитие предложенного в работе метода
в PyCharm и других средах разработки для динамически типизированных языков
способно качественно повысить уровень их поддержки, приблизив его к
возможностям, предлагемым инструментами для Java и C\#.





