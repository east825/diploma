%%%%%%%%%%%%%%%%%%%%%%%%%%%%%%%%%%%%%%%%%%%%%%%%%%%%%%%%%%%%%%%%%%%%%%%%%%%%%%%%
\conclusion
%%%%%%%%%%%%%%%%%%%%%%%%%%%%%%%%%%%%%%%%%%%%%%%%%%%%%%%%%%%%%%%%%%%%%%%%%%%%%%%%

Нами был предложен алгоритм для вывода типов параметров функций в программах на
Python, использующий информацию об атрибутах параметров, к которым происходит
обращение для подбора подходящих по интефейсу классов. 

Несмотря на высокую верхнюю границу сложности алгоритма по подбору классов,
статистика, собранная при помощи прототипа, продемонстрировала применимость
метода для анализа очень больших проектов, а сохранение требования
инкрементальности анализа позволяет использовать его в составе сред разработки. 

Хотя требование точности анализа достигнуто не было, благодаря
данным, собранным прототипом для нескольких десятков проектов с открытыми
исходными текстами, было найдено множество потенциальных направлений для
дальнейшего улучшения предлагаемого метода, большая часть которых, однако,
требует учета при анализе потоков управления и данных в программе. Например, 

\begin{enumerate}
    \item Использование только атрибутов, к которым происходит обращение на всех
      путях потока исполнения в теле функции.
    \item Учитывание псевдонимов (\emph{aliases}) параметров функций.
    \item Использовании информации о структурных типах параметров для вызываемых
      функций.
\end{enumerate}

По причине высокой трудоемкости и ограничений используемых в разработке
стандартных компонентов это не было сделано в прототипе, но в составе среды
разработки PyCharm, для которой разрабатывался алгоритм, имеются все средства
для этого. Кроме того, совмещение предложенного нами подхода с уже имеющимися
возможностями по выводу типов в PyCharm также должно значительно повысить
точность получаемых результатов.

Между тем, некоторая функциональность среды разработки, не требующая определения
единственного возможного типа, такая как автодополнение, может быть улучшена уже
при помощи текущей версии алгоритма.

Областями дальнейшего развития идеи, также представляющими интерес, но не
рассмотренными в данной работе, являются применения алгоритма для вывода типов
локальных переменных, а также использование его с другими динамически
типизированными языками, такими как Ruby и JavaScript.

Мы считаем, что применение и дальнейшее развитие предложенного в работе метода
в PyCharm и других средах разработки для динамически типизированных языков
способно качественно повысить уровень их поддержки, приблизив его к
возможностям, предлагемым инструментами для Java и C\#.





