%%%%%%%%%%%%%%%%%%%%%%%%%%%%%%%%%%%%%%%%%%%%%%%%%%%%%%%%%%%%%%%%%%%%%%%%%%%%%%%%
\intro
%%%%%%%%%%%%%%%%%%%%%%%%%%%%%%%%%%%%%%%%%%%%%%%%%%%%%%%%%%%%%%%%%%%%%%%%%%%%%%%%

Динамически типизироваными (\emph{dynamically typed}) называются языки
программирования, в которых все проверки типов осуществляются во время
исполнения программы. Примерами динамически типизированных языков являются
Python, Ruby и JavaScript.  Напротив, статически типизированными
(\emph{statically typed}) называются языки, в которых благодаря наличию системы
типов (\emph{type system}) оказывается возможным проверить корректность типов
выражения в программе до ее исполнения (статически), хотя для обеспечения
безопасности языка (\emph{type safety}) ряд проверок осуществляется и во время
исполнения.  В качестве примеров статически типизированных языков можно привести
Java, C\# и Haskell.

Несмотря на популярность динамически типизированных языков,
хорошо известны и проблемы, связанные с фактом проверки типов только во время
исполнения.

\begin{itemize}
  \item Ряд ошибок, например, передача аргумента неправильного типа при вызове
      функции или обращение к несуществующему методу объекта обнаруживается
      только во время исполнения программы и только тогда, когда через
      соответствующий участок программы проходит поток исполнения.

  \item Отсутствие явных аннотаций типов в тексте программы, например, для
      параметров функций без наличия хорошей документации усложняет ее
      сопровождение и изучение незнакомого API.

  \item Осуществляемые в процессе исполнения программы проверки типов,
      проявляющиеся, например, в поиске определения метода при каждом обращении
      к нему, приводят к существенному падению производительности
      интерпретаторов. Какие-либо предварительные оптимизации, которые
      должны быть подкреплены точной информацией о типах в программе, также
      оказываются невозможны. \todo{Возможно стоит посмотреть в статьях про
        упомниния о подобных оптимизациях}
\end{itemize}

Первая проблема отчасти решаются различными инструментами для статического
анализа программ, в частности, так называемыми, *Lint анализаторами (JSLint,
PyLint) и интегрированными средами разработки (IDE), в которых предпринимается
попытка вывести как можно больше информации о типах статически.  Однако на
практике даже в коммерческих продуктах в ряде случаев возможностей проводимого
анализа оказывается недостаточно.

Подобный анализ с целью вывода типов, как впрочем и любой другой вид
статического анализа, особенно сложен в IDE, в силу их тесной
интеграции с самим процессом разработки программ. В частности должна
поддерживаться работа с незавершенными программами, анализ должен быть быстрым,
а результаты как можно более точными, так как любое ложное сообщение об ошибке
является крайне нежелательным.

Анализ публикаций и существующих решений в области показал, что
многие из применяемых подходов для вывода типов в динамически типизированных языках либо
не адаптированы для современных высокоуровневых языков программирования, таких
как Python или Ruby, либо не удовлетворяют критериям применения в составе сред
разработки, требуя, например, использования специального инструментированного
интерпретатора или глобального анализа проекта, либо просто
обладают недостаточной точностью. Кроме того, практически все описанные в публикациях
подходы не формализованы, а детали реализации плохо описаны, что
затрудняет их использование где-либо еще помимо оригинальных работ. 

В данной работе проводится сравнение существующих подходов для
вывода типов типов в программах на динамически типизированных языках
применительно к языку Python. Также обсуждаются особенности использования этих
алгоритмов в составе интегрированных сред разработки.

Кроме того, нами описывается подход к анализу программ на Python, основанный на
структурной эквивалентности типов и использовании информации о передаваемых
объекту сообщениях для подбора подходящих по интерфейсу классов. Созданный нами
для анализа ряда существующих проектов прототип показал, что таким образом
оказывается возможным вывести типы в ряде сложных случаев, где другие методы
оказываются неэффективными, а достигнутые быстродействие и точность анализа
позволяют использовать предложенный подход в составе сред разработки, главным
образом, в IDE PyCharm\footnote{\url{http://www.jetbrains.com/pycharm/}}.

% С другой стороны, динамически типизированные языки предлагают более естественный и
% копактный, лишенный аннотаций синтаксис, ускорееный цикл разработки и богатые
% возможности метапрограммирования, подзволяющие, например, обойтись без этапа
% генерации кода, с помощью которого решается ряд проблемы в языках со статической
% типизацией. \todo{Сформулировать иначе. Данная работа адресована в первую очередь первым трем проблемам, так как они
% среды разработки позволяют

% Для того, чтобы дать определение динамически типизированным языкам, необходимо
% ввести понятие типа применительно к языкам программирования. Тип в языках
% программирования задает ограничения на возможные значения выражения в программе.
% Например, только целые числа в определенном диапазоне, ограниченном аппаратной
% платформой, массивы таких чисел или константы из определенного множества. В
% языках имеющих составные структуры данных тип также служит для описания их
% интерфейса, например, доступных полей структуры в программе C и методов объекта
% класса в Java. В еще более высокоуровневых языках позволяющих оперировать
% функциями, классами и модулями как полноценными объектами программы (объекты
% первого класса) типы появляются и у них.

% Несколько определений типов из литературы:

% \begin{quote}
% Type is a constraint which defines the set of valid values which
% conform to it. (L.~Tratt)
% \end{quote}

% \begin{quote}
% A program variable can assume a range of values during the execution of a
% program. An upper bound of such a range is called a type of the variable.
% (L.~Cardeli)
% \end{quote}


% Языки со статической типизацией позволяют обнаружить ряд ошибок связанных с
% типами, например, попытку обращения к несуществующему полю объекта или
% передачу в функцию числа неправильной разрядности, до запуска программы.
% Фактически статическая типизация является самым распространенным средством
% обеспечения качества ПО. С другой стороны, такие проверки не даются бесплатно.
% Во-первых, программу на явно типизированных языках (explicitly typed)
% приходится явно аннотировать для того, чтобы позволить компилятору проверить
% ее корректность. Причем, в ряде языков доля таких аннотаций в тексте программы
% может быть весьма значительной. Во-вторых, для обеспечения безопасности (type
% safety) программы существующие алгоритмы проверки типов могут отвергнуть и
% корректную программу. Например, ту, где ошибка находится в недостижимой участке.
% Таким образом, оказывается нельзя внести небольшое изменение в часть программы
% без необходимости изменять остальную. В целом естественный ход мыслей
% разработчика приходится искусственно укладывать в рамки системы типов языка.
% Причем ограничения последней могут быть весьма не очевидны. К другим недостаткам
% можно отнести увеличивающуюся сложность разработки языка и среды его
% исполнения, а также увеличивающиеся время перед запуском программы из-за
% проводимых проверок типов.

% Динамически типизированные языки отличаются тем, что проверки типов в них
% происходят во время исполнения программы. На этом этапе конкретные типы всех
% объектов в программе уже известны среде исполнения, и аннотации, вносимые
% программистом, оказываются не нужны.

% Определения из литературы:

% \begin{quote}
% A dynamic language is one that defers as many decisions as possible to runtime.
% (M.~Madsen/B.~Cannon)
% \end{quote}

% \begin{quote}
% Dynamic typing … ensures that no correct program execution is stopped
% prematurely — only programs about to “go wrong” at runtime are rejected.
% (M.~Furr)
% \end{quote}


% \begin{quote}
% Dynamic typing, at its simplest level, is when type checks are left until
% run-time. (L.~Tratt)
% \end{quote}

% \begin{quote}
% Dynamic languages, like Python, are attractive because they guarantee that no
% correct program is rejected prematurely. (E.~Maya)
% \end{quote}

% \begin{quote}
% Python is “dynamically typed” language because, in general, the type of any
% variable is not known definitively until run-time. (J.~Aycock)
% \end{quote}

% Динамически типизированные языки часто предоставляют программисту большие
% выразительные возможности, более естественный синтаксис, и ускоренный цикл
% разработки, однако страдают от недостатков, вызванных отсутствием статических
% проверок. Многие ошибки оказываются не обнаруженными до запуска программы, и
% могут остаться незамеченными в том числе во время ее работы, если через участок,
% содержащий ошибку, не проходит поток исполнения. Поскольку аннотации типов
% играют также и роль простейшей документации программы, в их отсутствии
% усложняется процесс поддержки и сопровождения приложения.  Без информации о
% типах болле трудоемкой оказывается и поддержка языка инструментами разработки, в
% частности интегрированными средами разработки (IDE). Проведение проверок типов,
% например, разрешения атрибутов объекта во время исполнения программы негативно
% сказывается на ее производительности. В то же время отсутствие информации о
% типах на этапе компиляции программы не позволяет выполнить каких-либо
% оптимизаций для повышения ее быстродействия.

% Между тем, некоторые существующие языки со статической типизацией, главным
% образом, функциональные, например, Haskell или Scala, позволяют сохранить
% преимущества статической типизации, предоставляя синтаксис, лишенный
% значительной части явных аннотаций типов. Это возможно благодаря алгоритмам
% вывода типов (type inference), которые заменяет проверку того, что указанные
% программистом типы корректны, доказательством того, что такая комбинация типов
% в принципе существует. Такие языки называются также неявно типизированными
% (implicitly typed).

% Следует отметить, что язык должен изначально проектироваться с учетом
% возможности такого вывода, и ряд особенностей динамически типизированных
% языков, обсуждаемых в последующих разделах, не позволяет осуществить такой
% вывод для произвольной программы, на них написанной.

