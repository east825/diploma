% vim: fenc=utf-8

\documentclass{beamer}

% Should not be used with xelatex
% \usepackage[utf8x]{inputenc}
\usepackage[russian]{babel}
\usepackage{minted}
\ifxetex
  \usepackage{fontspec}
  \defaultfontfeatures{Ligatures=TeX} % To support LaTeX quoting style
  \setmainfont{Times New Roman}
  \setmonofont{Consolas}
\else
  \usepackage[T1]{fontenc}
  \usepackage[utf8]{inputenc}
\fi

% Beamer settings
\mode<presentation>
{
  % Variant #1
  \usetheme{Madrid}
  \usecolortheme{Seagull}
  % \usecolortheme{Spruce}

  % Variant #2
  % \usetheme{EastLansing}
  % \useinnertheme{rounded}

  \usefonttheme{serif}
  \setbeamertemplate{navigation symbols}{}
  \setbeamertemplate{itemize item}{$\Box$}
  % \setbeamertemplate{title page}
  % {
    % \begin{centering}
      % \insertinstitute
      % \vfill
      % \begin{beamercolorbox}[sep=8pt,center]{title}
        % {\usebeamerfont{title}\usebeamerfont{title}\inserttitle}
      % \end{beamercolorbox}
    % \end{centering}
    % \vfill
  % }
} 


% Minted settings
\usemintedstyle{trac}
% \definecolor{bg}{rgb}{0.10, 0.10, 0.10}
\newminted{python}{fontsize=\scriptsize}

% Miscellaneous customization
 

% Header and metainfo
\title[Вывод типов]{Вывод типов для динамически типизированного языка в 
интегрированной среде разработки}
\author{Голубев Михаил}
\institute[ИИТУ]{
  Санкт-Петербургский государственный политехнический университет \\
  Институт информационных технологий и управления \\
  Кафедра компьютерных систем и программных технологий
}
\date{октябрь 2013}

%%%%%%%%%%%%%%%%%%%%%%%%%%%%%%%%%%%%%%%%%%%%%%%%%%%%%%%%%%%%%%%%%%%%%%%%%%%%%%%%

\begin{document}

% Титульный слайд
\begin{frame}
  \titlepage
\end{frame}


% Постановка задачи
\begin{frame}{Постановка задачи}
  \begin{block}{Задача}
    Разработать систему вывода типов в средах разработки для динамически типизированных языков.
    \footnotemark[1]
  \end{block}
  \footnotetext[1]{Первоначальная реализация для языка Python и платформы IntelliJ (PyCharm)}
\end{frame}


% Проблемы и ограничения
\begin{frame}{Проблемы и ограничения}
  Анализ
  \begin{itemize}
    \item должен выполняться быстро \emph{(анализ “на лету” vs. пакетные анализаторы)}
    \item должен быть работать с незавершенными/частично корректными программами \emph{(как и всегда для сред разработки)}
    \item точность результатов важнее полноты \emph{(потому что все ошибки, все равно не найти)}
  \end{itemize}

  Основная сложность --- принимаемые/возвращаемые значения функций 
  ($\Rightarrow$ нелокальный анализ), а также динамические “трюки”: \texttt{monkey-patching}, 
  \texttt{descriptors}, \texttt{import-hooks}, \texttt{exec/eval}, различные средства интроспекции и пр.
\end{frame}


\begin{frame}[fragile]{Пример}
Программа на Python + PyQt4

\begin{pythoncode}
import os
import son
from PyQt4.QtCore import QSettings

if __name__ == '__main__':
    s = QSettings()
    path = s.value('ConfigPath', '~/.settings')
    path = os.path.expanduser(path)
    with open(path, 'r') as f:
        config = json.load(f)
\end{pythoncode}
\end{frame}


\begin{frame}[fragile]{Пример (продолжение)}{Результат запуска}
\begin{pythoncode}
Traceback (most recent call last):
File "C: /Users/Mikhail/home/development/repos/diploma/samples/errors/qsettings.py", line 10, in <module>
path = os.path.expanduser(path)
File "C:\Python27\lib\ntpath.py", line 279, in expanduser
if path[:1] != '~':
TypeError: 'QVariant' object has no attribute '__getitem__'
\end{pythoncode}
\end{frame}


\begin{frame}[fragile]{Пример (продолжение)}{Исправленная версия}
\begin{pythoncode}
import os
import json
from PyQt4.QtCore import QSettings
if __name__ == '__main__':
   s = QSettings()
   path = s.value('ConfigPath', '~/.settings').toString()
   path = os.path.expanduser(path)
   with open(path, 'r') as f:
       config = json.load(f)
\end{pythoncode}
\end{frame}


\begin{frame}[fragile]{Пример (продолжение)}{Результат запуска}
\begin{pythoncode}
Traceback (most recent call last): 
File "C: /Users/Mikhail/home/development/repos/diploma/samples/errors/qsettings.
py", line 10, in <module>
 path = os.path.expanduser(path)
 File "C:\Python27\lib\ntpath.py", line 282, in expanduser
 while i < n and path[i] not in '/\\':
TypeError: 'in <string>' requires string as left operand, not QString
\end{pythoncode}
\end{frame}


\begin{frame}[fragile]{Пример (продолжение)}{Итоговая версия}
\begin{pythoncode}
import os
import json
from PyQt4.QtCore import QSettings
if __name__ == '__main__':
    s = QSettings()
    path = unicode(s.value('ConfigPath', '~/.settings').toString() )
    path = os.path.expanduser(path)
    with open(path, 'r') as f:
        config = json.load(f)
\end{pythoncode}

Теперь все работает\ldots

\end{frame}

\begin{frame}{Текущее состояние}
\begin{itemize}
  \item Поиск литературы \\
  \small{\emph{Много материалов про вывод типов в целом, не так много про вывод в динамически типизированных языках}}

  \item Анализ существующих решений
    \begin{itemize}
      \item Подходы к работе с типами в PyCharm и других средах JetBrains: 
      RubyMine, PhpStrom, WebStorm
      \item Инструменты PyFlakes, PyLint
    \end{itemize}
\end{itemize}
\end{frame}
\end{document}
